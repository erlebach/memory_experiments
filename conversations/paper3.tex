\documentclass[11pt]{article}
\usepackage[utf8]{inputenc}
\usepackage[T1]{fontenc}
\usepackage{amsmath}
\usepackage{amsfonts}
\usepackage{amssymb}
\usepackage{graphicx}
\usepackage{booktabs}
\usepackage{array}
\usepackage{multirow}
\usepackage{longtable}
\usepackage{url}
\usepackage{cite}
\usepackage{algorithm}
\usepackage{algorithmic}
%\usepackage[margin=1in]{geometry}
\usepackage[top=1in, bottom=1in, left=1in, right=1in]{geometry}
%\usepackage{showframe}
\usepackage{natbib}
\usepackage{hyperref}

% Custom commands
\newcommand{\bm}[1]{\boldsymbol{#1}}
\newcommand{\norm}[1]{\left\|#1\right\|}

\title{A Unified State-Space Memory Framework: Bridging Neural Sequence Models and Human Episodic Memory}

\author{
Gordon Erlebacher
}

\date{2025-06-21}

\begin{document}
\maketitle

\begin{abstract}
We introduce a general state-space model (SSM) framework that unifies architectures traditionally viewed as disparate: human memory models such as Context Maintenance and Retrieval (CMR), and modern neural sequence models including RetNet and Mamba. Despite differences in their origin and application domains, we show that all can be expressed as recurrent or continuous-time SSMs with input-dependent dynamics. We formalize this connection, demonstrate how ideas such as surprise-based gating and event segmentation naturally emerge within this framework, and propose new hybrid models that combine interpretability and performance. We present benchmark results across cognitive modeling, language modeling, and memory-intensive reasoning tasks, highlighting the trade-offs and synergies among the different instantiations.
\end{abstract}

\section{Introduction}
Memory is fundamental to both cognition and sequence modeling. While cognitive science has deve  loped sophisticated models of human episodic memory such as Context Maintenance and Retrieval (  CMR), the deep learning community has independently created powerful sequence models like RetNe  t and Mamba. Additionally, neuro-symbolic hybrid approaches like Titan have emerged that bridge   these domains. Despite their different origins and applications, these architectures share rem  arkably similar core dynamics.

In this work, we propose a unifying framework that reveals these diverse approaches as points i  n a broader space of state-based memory models. Our contributions include: (1) a general state-  space formulation that encompasses existing memory architectures, (2) a flexible implementation   framework supporting plug-and-play memory modules, and (3) comprehensive benchmarks across cog  nitive modeling, language modeling, and reasoning tasks.


\section{Related Work}

\subsection{Memory Models in Cognitive Science}
Howard and Kahana~\cite{howard2002temporal} introduced the Temporal Context Model, which was extended by Polyn et al.~\cite{polyn2009context} into CMR with context reinstatement mechanisms. Gershman et al.~\cite{gershman2017bayesian} developed Bayesian approaches to event segmentation that inform our surprise-based gating.

\subsection{Neural State-Space Models}
Gu et al.~\cite{gu2021s4} introduced Structured State Spaces (S4), while Dao et al.~\cite{dao2023mamba} developed Mamba and Mamba-2 with selective SSM implementations. Sun et al.~\cite{sun2023retnet} proposed Retentive Networks as an alternative to attention mechanisms.


\subsection{Memory-Augmented Neural Networks}
Graves et al.~\cite{graves2016dnc} introduced Differentiable Neural Computers, and Rae et al.~\cite{rae2020compressive} developed Compressive Transformers. Our Titan model builds on surprise-modulated sparse memory approaches.


\subsection{Surprise and Adaptive Gating in Memory}
Gershman~\cite{gershman2019prediction} connected prediction error to memory encoding, while Ritter et al.~\cite{ritter2018meta} explored meta-learning with surprise-based memory. Goyal et al.~\cite{goyal2022retrieval} investigated retrieval-augmented models with adaptive gating.


\section{A General State-Space Framework for Memory}
adf

% (Proposed new subsection)**
\subsection{Formalism and Core Components (NEW)} 
We define a general state-space memory framework where the context state $c_i$ evolves according to:
\begin{equation}
c_{i} = A(f_i) c_{i-1} + B(f_i)
\end{equation}

or in continuous time:
\begin{equation}
\frac{dx}{dt} = A(x, u) x + B(x, u)
\end{equation}

The gate formulation introduces surprise-modulated updates:
\begin{equation}
c_i = \rho_i c_{i-1} + \beta_i \cdot c^{\text{IN}}_i, \quad \beta_i = g(\text{surprise}_i)
\end{equation}

where the parameters $A$, $B$, and $\beta$ can be fixed, learned, or conditioned on input or state. This formulation captures the essential dynamics of memory systems while allowing for diverse instantiations.

% (Proposed new subsection)**
\subsection{Emergent Properties: Event Segmentation and Surprise (NEW)}
adf 


\section{Mapping Existing Architectures to the Framework}
adf

\subsection{Cognitive Models: Context Maintenance and Retrieval (CMR)}
CMR operates in discrete time with psychologically interpretable dynamics. It employs surprise-dependent updates where:
\begin{equation}
\beta_i = g\left(\left\|\nabla \mathcal{L}_{\text{memory}}\right\|\right)
\end{equation}

The context serves as an internal state encoding both temporal and source features, enabling the model to capture human-like memory phenomena such as serial position effects and temporal clustering.


\subsection{Neural Sequence Models: RetNet}
RetNet implements low-rank state evolution with implicit recurrence and exponential decay. Its linear memory complexity makes it efficient for long sequences while maintaining the expressiveness needed for language modeling tasks.

\subsection{Structured SSMs: Mamba and Mamba-2}
These continuous-time state-space models feature input-conditioned dynamics $A(x)$ and achieve fast inference through selective SSM implementations. The input-dependent transition matrices allow for adaptive processing of sequential information.

\subsection{Titan}
Titan uses surprise signals (e.g., prediction error) to modulate memory access and writing, encouraging sparse and interpretable memory updates. This approach bridges the gap between cognitive plausibility and computational efficiency.



\section{Proposed Implementations and Hybrid Models}
We develop a flexible PyTorch framework implementing the general update equation with plugin modules for:
\begin{itemize}
\item Input-conditioned $A(f_i)$ matrices
\item Surprise-based gating mechanisms  
\item Low-rank or diagonal state transitions
\item Learned versus interpretable memory representations
\end{itemize}

The framework supports both series and parallel composition of memory blocks, enabling exploration of hybrid architectures that combine the strengths of different approaches.


\subsection{Synergistic Architectures (NEW)} % (Proposed new subsection)**
adf

\subsection{LoRA-Enhanced Memory Modules}
We enhance pretrained models with memory capabilities using Low-Rank Adaptation (LoRA), preserving base model performance while adding episodic functionality.

\subsubsection{Two-Stage Training Protocol}
\textbf{Phase 1: Offline Fine-Tuning (Optional)}
\begin{itemize}
\item
LoRA adapters in memory blocks trained on multi-task datasets
\item
Pretrained base remains frozen
\item
Memory modules learn task-specific dynamics and gating functions
\end{itemize}

\textbf{Phase 2: Inference-Time Adaptation}
\begin{itemize}
\item
Memory adapters continue evolving during deployment
\item
Gradual learning rate scheduling based on surprise signals
\item
Runtime input serves as training data for continuous adaptation
\end{itemize}

\subsubsection{Inference-Time Adaptation}
We introduce specialized protocols for evaluating adaptive memory:
\begin{itemize}
\item \textbf{Static vs. Adaptive Comparisons}: Performance differences over time
\item \textbf{Episodic Reinstatement}: Ability to retrieve context-relevant items
\item \textbf{Temporal Generalization}: Accuracy as function of time and updates
\item \textbf{Catastrophic Interference}: Performance preservation across task switches
\end{itemize}

\section{Experimental Setup}
adf

\subsection{Evaluation Suite and Tasks}
adf

\subsubsection{Synthetic Free Recall Task (CMR-Inspired)}
\textbf{Goal}: Test memory reinstatement and temporal clustering.

Setup : Present a sequence of tokens (e.g., 30), followed by a recall prompt.

Metrics :
\begin{itemize}
\item Serial position effect (primacy/recency)
\item Lag-CRP (Conditional Response Probability by temporal lag)
\item Source clustering (if tasks switch mid-list)
\end{itemize}

Use : Validate cognitive plausibility and long-range associative binding.

\subsubsection{Copy-and-Repeat Task}
Goal : Test capacity and fidelity of memory.

Setup : Input a sequence (e.g., 1024 symbols), followed by a repeat instruction.

Metrics : Exact match accuracy, copy length limit

Use : Assess ability to maintain precise sequence over long span (used in S4, RNN, Mamba papers).

\subsubsection{Long Range Arena (LRA) Subset}
Goal : Evaluate efficient long-sequence modeling.

Subtasks :
\begin{itemize}
\item ListOps (hierarchical reasoning)
\item Text (semantic classification)
\item Retrieval (explicit memory retrieval)
\end{itemize}

Metrics : Accuracy, speed, memory usage

Use : Benchmark against state-of-the-art long-sequence models.

\subsubsection{Event Segmentation}
Goal : Test the model’s ability to learn event boundaries.

Setup : Input consists of mini-scenes (e.g., toy videos, or token streams from synthetic storylets) with boundary transitions.

Labels : Predict segment boundaries.

Use : Evaluate surprise-driven gating and boundary detection (aligns with gradient-based $\beta$)

\subsubsection{Continual Learning with Interleaved Tasks}
Goal : Test interference and transfer.

Setup : Present multiple tasks (e.g., arithmetic, classification, reasoning) in a time-ordered stream.

Metrics : Task accuracy over time, forgetting rate

Use : Measure memory modularity, resistance to catastrophic forgetting.

\subsubsection{Question Answering on Structured Documents}
Goal : Evaluate retrieval + reasoning.

Setup : Feed long document (e.g., 3K tokens) with embedded Q\&A targets.

Metrics : QA accuracy, retrieval score, compression robustness

Use : Test models’ ability to localize and abstract memory.

\subsubsection{Memory Generalization Task}
Goal : Test interpolation and extrapolation from stored memory.

Setup : Store N (query, answer) pairs; test with novel interpolated or perturbed querie        s.

Metrics : Embedding similarity, generalization accuracy

Use : Evaluate compositional or non-local generalization capacity.

\subsubsection{Surprise-Modulated Recall}
Goal : Explicitly probe  $\beta$  (gating) dynamics.

Setup : Interleave predictable and unpredictable events.

Labels : Predict whether memory updated or not.

Use : Validate gating mechanisms like  $\beta(\text{surprise})$.


\subsection{Metrics and Objectives}
adf

%    * *(Moves "Metrics and Loss Functions" here.)*
\subsubsection{Loss Functions}
We employ multiple loss functions depending on the task requirements:

\textbf{Cross-Entropy Loss} for classification and prediction:
\begin{equation}
\mathcal{L}_{\text{CE}} = -\sum_{i} y_i \log \hat{y}_i
\end{equation}

\textbf{Mean Squared Error} for memory reconstruction:
\begin{equation}
\mathcal{L}_{\text{MSE}} = \frac{1}{N} \sum_{i=1}^N \left\|\hat{f}_i - f_i\right\|^2
\end{equation}

\textbf{Surprise Signal} from gradient norm:
\begin{equation}
\text{surprise}_i = \left\|\nabla_{c_{i-1}} \mathcal{L}_{\text{memory}}(f_i, c_{i-1})\right\|
\end{equation}


\subsubsection{Behavioral Metrics}
Beyond standard accuracy measures, we track memory-specific behaviors:
\begin{itemize}
\item \textbf{Serial Position Curve}: Accuracy vs. item position (primacy/recency effects)
\item \textbf{Lag-CRP}: Probability of recalling item $i+k$ given item $i$ was recalled
\item \textbf{Source Clustering}: Successive recall of items from the same context
\item \textbf{Memory Utilization}: Active usage and retention patterns
\end{itemize}


\subsubsection{Composite Objective}
For multi-task scenarios, we use weighted combinations:
\begin{equation}
\mathcal{L}_{\text{total}} = \lambda_{\text{task}} \mathcal{L}_{\text{CE}} + \lambda_{\text{recall}} \mathcal{L}_{\text{memory}} + \lambda_{\text{gating}} \mathcal{L}_{\text{entropy}}
\end{equation}

\subsection{Training and Stabilization}
Dynamic memory updates introduce unique convergence challenges.

% *(Moves "Training Dynamics and Stabilization" here.)*
\subsubsection{Stabilization Techniques}
%
\begin{table}[h]
\centering
\begin{tabular}{@{}ll@{}}
\toprule
\textbf{Problem} & \textbf{Solution} \\
\midrule
Vanishing memory updates & Min-gate floor, gate warm-up \\
Undertrained memory parameters & Auxiliary recall loss \\
Gradient shocks from surprises & Smooth gating functions \\
Mamba matrix instability & Spectral norm, low-rank $A(x)$ \\
Gradient starvation & Residual bypass paths \\
\bottomrule
\end{tabular}
\caption{Stabilization techniques for dynamic memory training}
\label{tab:stabilization}
\end{table}

\subsubsection{The Role of Residual Connections}
To mitigate gradient starvation when gating values approach zero, we introduce residual bypass paths:
\begin{equation}
c_i = (1 - \beta_i) c_{i-1} + \beta_i \cdot c^{\text{IN}}_i + \alpha \cdot \text{Res}(f_i)
\end{equation}

This ensures non-zero gradient flow even when $\beta_i \to 0$, allowing the system to begin learni before gating becomes informative.
                  
\section{Results and Analysis}
This section presents the outcomes of evaluating our proposed framework on the benchmarks and tasks defined in the experimental setup. We analyze performance across different memory models, highlighting key trade-offs and synergies.

\subsection{Performance on Benchmark Tasks}
We evaluate our framework across diverse tasks that probe different aspects of memory functionality. The results on standard benchmarks are summarized in Table~\ref{tab:benchmarks}.

\begin{table}[h]
\centering
\begin{tabular}{@{}lll@{}}
\toprule
\textbf{Task Type} & \textbf{Dataset} & \textbf{Metric} \\
\midrule
Human memory modeling & PEERS free recall & Serial position, CRP \\
Language modeling & PG19, Wikitext-103 & Perplexity \\
Long-context reasoning & Long Range Arena & Accuracy, efficiency \\
Event segmentation & BABI, CLEVR-Humans & F1, segmentation error \\
Continual learning & Permuted MNIST & Forgetting, accuracy \\
\bottomrule
\end{tabular}
\caption{Evaluation benchmarks across different task categories.}
\label{tab:benchmarks}
\end{table}

\subsection{Analysis of Trade-offs and Synergies}
Here, we analyze the results to highlight trade-offs (e.g., computational cost vs. cognitive plausibility) and synergies (e.g., how hybrid models outperform their individual components). This subsection will contain figures and analysis comparing CMR, Mamba, RetNet, and hybrid models on the evaluation suite. For example, plots of Lag-CRP curves, Perplexity scores, and LRA accuracy.

\subsection{Ablation Studies}
To understand the contribution of each component in our framework, we conduct a series of ablation studies. We systematically remove or alter key mechanisms, such as surprise-based gating and LoRA-based adaptation, and measure the impact on performance. This subsection will contain tables showing the results of ablation studies, e.g., model performance with and without surprise gating.


\section{Applications}
Our unified framework enables several novel applications:

\begin{itemize}
\item \textbf{Interpretable memory in AI}: CMR-inspired structure provides cognitive plausibility
\item \textbf{Efficient long-range dependency modeling}: RetNet and Mamba optimizations
\item \textbf{Meta-cognitive control}: Surprise gating for online adaptation
\item \textbf{Hybrid systems}: Applications in education, cognitive modeling, and LLM enhancement
\end{itemize}
                        
\section{Discussion}
The proposed framework, which enhances pretrained models with LoRA-based
memory modules, introduces a novel paradigm for creating adaptive and
efficient AI systems. This section analyzes the advantages and challenges
of this approach and situates it within the context of existing literature.

\subsection{Advantages of the Proposed Framework}

\subsubsection{Parameter Efficiency and Modularity}
The use of \textbf{LoRA adapters} allows for the integration of memory
functionality without the need to update the large pretrained backbone.
This makes memory modules easy to switch, modify, or upgrade without
retraining the base model, which is ideal for multi-task and plug-and-play
deployments.

\subsubsection{Safe and Gradual Integration}
Initializing LoRA weights to \textbf{zero} ensures that the model's initial
behavior is identical to that of the base model. This strategy prevents the
destabilization of pretrained knowledge and allows for a gradual,
interpretable adaptation process driven entirely by new data.

\subsubsection{Inference-Time Personalization and Adaptation}
The framework enables the model to become \textbf{situationally adaptive},
tuning its memory pathways to the current context, user history, or
conversational domain. This is particularly valuable for applications in
conversational AI, lifelong learning, and domain adaptation where
offline retraining is impractical.

\subsubsection{Alignment with Cognitive Models}
By allowing memory to adapt during inference, the model mirrors key
\textbf{episodic memory dynamics} observed in human cognition, such as
context drift, memory reinstatement, and surprise-modulated updating.
This provides a bridge between high-performance neural models and
cognitively plausible architectures.

\subsection{Challenges and Considerations}

\subsubsection{Ill-Defined Loss at Inference}
A primary challenge is the lack of explicit supervision during
inference-time learning. To overcome this, the model may need to rely on
self-supervised or contrastive loss functions, or surrogate objectives
such as consistency, retrieval accuracy, or semantic alignment.

\subsubsection{Risk of Overfitting and Interference}
Continuous online updates to LoRA adapters risk overfitting to transient
or noisy inputs, potentially leading to catastrophic forgetting within
the memory modules. This necessitates mitigation strategies such as
intelligent gating mechanisms, surprise-aware learning rate schedules,
and sparsity constraints.

\subsubsection{Evaluation Complexity}
Evaluating an adaptive system requires more than static accuracy metrics.
It demands dynamic, lifespan-aware evaluation protocols that can track
memory quality, interference, and adaptation over time and across
changing contexts.

\subsubsection{Infrastructural Overhead}
Online adaptation introduces infrastructural demands, including the need
for gradient tracking at inference time, persistent state buffers for
memory traces and LoRA weights, and careful memory management on
deployment hardware.

\subsection{Relation to Existing Work}
Our approach synthesizes ideas from several distinct areas of machine
learning research. Table~\ref{tab:discussion_related_work} situates our
proposal relative to key existing works. The novel contribution of our
framework is the integration of these concepts into a unified,
memory-centric architecture.

\begin{table}[h]
\centering
\caption{Connections to Existing Research}
\label{tab:discussion_related_work}
\begin{tabular}{p{0.25\textwidth} p{0.35\textwidth} p{0.35\textwidth}}
\toprule
\textbf{Related Work} & \textbf{Key Contribution} & \textbf{Relation to Our Proposal} \\
\midrule
\textbf{LoRA (Hu et al., 2021)} & Low-rank adaptation for frozen models & We use LoRA specifically for trainable memory adapters. \\
\textbf{QLoRA (Dettmers et al., 2023)} & Quantized LoRA for memory efficiency & We can apply QLoRA to reduce the cost of memory modules. \\
\textbf{MetaICL (Kiyono et al., 2022)} & Few-shot inference-time adaptation & We share the goal of online adaptation but focus on explicit memory modules. \\
\textbf{Memory-Augmented Transformers (e.g., RETRO)} & Augmentation via external, non-trainable memory & Our memory is internal, modular, and dynamically trainable at inference time. \\
\textbf{Titan (internal)} & Surprise-modulated sparse memory updates & Our surprise-gated \(\beta\) is conceptually aligned with Titan's control mechanism. \\
\bottomrule
\end{tabular}
\end{table}

This combination of a frozen base, LoRA-injected memory, online gating,
and episodic dynamics represents a novel hybrid of neuroscience-inspired
modeling and parameter-efficient transformer tuning that, to our
knowledge, has not been published as a unified framework.
                          
\section{Conclusion and Future Work}
We have presented a unified state-space framework that bridges cognitive memory models and modern neural architectures. Our key contributions include:

\begin{itemize}
\item A general mathematical formulation encompassing diverse memory systems
\item Flexible implementation supporting hybrid architectures
\item Comprehensive evaluation across cognitive and computational tasks
\item Novel LoRA-based adaptation for inference-time memory learning
\end{itemize}

Future directions include extending to online learning scenarios, developing lifelong memory capabilities, and unifying with graph-based memory models. The framework opens new possibilities for interpretable AI systems that combine cognitive plausibility with computational efficiency.
               
\section*{Acknowledgments}
We thank the reviewers for their constructive feedback and suggestions for improving this work.
                    
% \section*{References}
\bibliographystyle{unsrt}
\begin{thebibliography}{20}

\bibitem{howard2002temporal}
M.~W. Howard and M.~J. Kahana.
\newblock A distributed representation of temporal context.
\newblock {\em Journal of Mathematical Psychology}, 46(3):269--299, 2002.

\bibitem{polyn2009context}
S.~M. Polyn, K.~A. Norman, and M.~J. Kahana.
\newblock A context maintenance and retrieval model of organizational processes in free recall.
\newblock {\em Psychological Review}, 116(1):129--156, 2009.

\bibitem{gershman2017bayesian}
S.~J. Gershman, Y.~Monfils, K.~A. Norman, and Y.~Niv.
\newblock The computational nature of memory modification.
\newblock {\em eLife}, 6:e23763, 2017.

\bibitem{gu2021s4}
A.~Gu, K.~Goel, and C.~R{\'e}.
\newblock Efficiently modeling long sequences with structured state spaces.
\newblock In {\em International Conference on Learning Representations}, 2022.

\bibitem{dao2023mamba}
T.~Dao, A.~Gu, M.~Eichhorn, A.~Rudra, and C.~R{\'e}.
\newblock Mamba: Linear-time sequence modeling with selective state spaces.
\newblock {\em arXiv preprint arXiv:2312.00752}, 2023.

\bibitem{sun2023retnet}
Y.~Sun, L.~Dong, B.~Patra, S.~Ma, S.~Huang, A.~Çelikyilmaz, H.~Wang, L.~Wang, S.~Liu, J.~Gao, and F.~Wei.
\newblock Retentive network: A successor to transformer for large language models.
\newblock {\em arXiv preprint arXiv:2307.08621}, 2023.

\bibitem{graves2016dnc}
A.~Graves, G.~Wayne, M.~Reynolds, T.~Harley, I.~Danihelka, A.~Grabska-Barwińska, S.~G. Colmenarejo, E.~Grefenstette, T.~Ramalho, J.~Agapiou, et~al.
\newblock Hybrid computing using a neural network with dynamic external memory.
\newblock {\em Nature}, 538(7626):471--476, 2016.

\bibitem{rae2020compressive}
J.~W. Rae, A.~Potapenko, S.~M. Jayakumar, C.~Hillier, and T.~P. Lillicrap.
\newblock Compressive transformers for long-range sequence modelling.
\newblock In {\em International Conference on Learning Representations}, 2020.

\bibitem{gershman2019prediction}
S.~J. Gershman.
\newblock The dopamine prediction error hypothesis: contributions to associative learning, memory, and attention.
\newblock {\em Annual Review of Psychology}, 70:379--402, 2019.

\bibitem{ritter2018meta}
S.~Ritter, J.~X. Wang, Z.~Kurth-Nelson, S.~M. Jayakumar, C.~Blundell, R.~Pascanu, and M.~Botvinick.
\newblock Been there, done that: Meta-learning with episodic recall.
\newblock In {\em International Conference on Machine Learning}, 2018.

\bibitem{goyal2022retrieval}
A.~Goyal, A.~Lamb, Y.~Bengio, and B.~van den Oord.
\newblock Object files and schemata: Factorizing declarative and procedural knowledge in dynamical systems.
\newblock {\em arXiv preprint arXiv:2006.16225}, 2022.

\end{thebibliography}

\section*{Appendix A}
adfl;k jadfa

\section*{Appendix B}
adfl;k jadfa
                          

\end{document}
